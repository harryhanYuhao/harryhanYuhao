\documentclass[10pt, letterpaper]{article}

% Packages:
\usepackage[
    ignoreheadfoot, % set margins without considering header and footer
    top=1.5 cm, % seperation between body and page edge from the top
    bottom=1.2 cm, % seperation between body and page edge from the bottom
    left=2 cm, % seperation between body and page edge from the left
    right=2 cm, % seperation between body and page edge from the right
    footskip=1.0 cm, % seperation between body and footer
    % showframe % for debugging 
]{geometry} % for adjusting page geometry
\usepackage{titlesec} % for customizing section titles
\usepackage{tabularx} % for making tables with fixed width columns
\usepackage{array} % tabularx requires this
\usepackage{xcolor} % for coloring text
\usepackage{enumitem} % for customizing lists
\usepackage{fontawesome5} % for using icons
\usepackage{amsmath} % for math
% \usepackage[
%     colorlinks=false,
% 	pdfborder={0 0 0}, % hyperlink borders will be invisible
%     pdfborderstyle={/S/U/W 1}, % border style will be underline of width 1pt
%     % pdftitle={Harry Han's CV},
%     % pdfauthor={Harry Han},
%     linkbordercolor=black,% hyperlink borders will be red
% ]{hyperref} % for links, metadata and bookmarks
\usepackage[hidelinks]{hyperref}
\usepackage[pscoord]{eso-pic} % for floating text on the page
\usepackage{calc} % for calculating lengths
\usepackage{bookmark} % for bookmarks
\usepackage{lastpage} % for getting the total number of pages
\usepackage{changepage} % for one column entries (adjustwidth environment)
\usepackage{paracol} % for two and three column entries
\usepackage{ifthen} % for conditional statements
\usepackage{needspace} % for avoiding page brake right after the section title
\usepackage{iftex} % check if engine is pdflatex, xetex or luatex

% Ensure that generate pdf is machine readable/ATS parsable:
\ifPDFTeX
    \input{glyphtounicode}
    \pdfgentounicode=1
    \usepackage[T1]{fontenc}
    \usepackage[utf8]{inputenc}
    \usepackage{lmodern}
\fi

\usepackage{charter}

% Some settings:
\raggedright
\AtBeginEnvironment{adjustwidth}{\partopsep0pt} % remove space before adjustwidth environment
\pagestyle{empty} % no header or footer
\setcounter{secnumdepth}{0} % no section numbering
\setlength{\parindent}{0pt} % no indentation
\setlength{\topskip}{0pt} % no top skip
\setlength{\columnsep}{0.15cm} % set column seperation
\pagenumbering{gobble} % no page numbering

\titleformat{\section}{\needspace{4\baselineskip}\bfseries\large}{}{0pt}{}[\vspace{1pt}\titlerule]

\titlespacing{\section}{
    % left space:
    -1pt
}{
    % top space:
    0.3 cm
}{
    % bottom space:
    0.2 cm
} % section title spacing

\renewcommand\labelitemi{$\vcenter{\hbox{\small$\bullet$}}$} % custom bullet points
\newenvironment{highlights}{
    \begin{itemize}[
        topsep=0.10 cm,
        parsep=0.10 cm,
        partopsep=0pt,
        itemsep=0pt,
        leftmargin=0 cm + 10pt
    ]
}{
    \end{itemize}
} % new environment for highlights


\newenvironment{highlightsforbulletentries}{
    \begin{itemize}[
        topsep=0.10 cm,
        parsep=0.10 cm,
        partopsep=0pt,
        itemsep=0pt,
        leftmargin=10pt
    ]
}{
    \end{itemize}
} % new environment for highlights for bullet entries

\newenvironment{onecolentry}{
    \begin{adjustwidth}{
        0 cm + 0.00001 cm
    }{
        0 cm + 0.00001 cm
    }
}{
    \end{adjustwidth}
} % new environment for one column entries

\newenvironment{twocolentry}[2][]{
    \onecolentry
    \def\secondColumn{#2}
    \setcolumnwidth{\fill, 4.5 cm}
    \begin{paracol}{2}
}{
    \switchcolumn \raggedleft \secondColumn
    \end{paracol}
    \endonecolentry
} % new environment for two column entries

\newenvironment{threecolentry}[3][]{
    \onecolentry
    \def\thirdColumn{#3}
    \setcolumnwidth{, \fill, 4.5 cm}
    \begin{paracol}{3}
    {\raggedright #2} \switchcolumn
}{
    \switchcolumn \raggedleft \thirdColumn
    \end{paracol}
    \endonecolentry
} % new environment for three column entries

\newenvironment{header}{
    \setlength{\topsep}{0pt}\par\kern\topsep\centering\linespread{1.5}
}{
    \par\kern\topsep
} % new environment for the header

\newcommand{\placelastupdatedtext}{% \placetextbox{<horizontal pos>}{<vertical pos>}{<stuff>}
  \AddToShipoutPictureFG*{% Add <stuff> to current page foreground
    \put(
        \LenToUnit{\paperwidth-2 cm-0 cm+0.05cm},
        \LenToUnit{\paperheight-1.0 cm}
    ){\vtop{{\null}\makebox[0pt][c]{
        \small\color{gray}\textit{Last updated in September 2024}\hspace{\widthof{Last updated in September 2024}}
    }}}%
  }%
}%

% save the original href command in a new command:


% new command for external links:


\begin{document}
    \newcommand{\AND}{\unskip
        \cleaders\copy\ANDbox\hskip\wd\ANDbox
        \ignorespaces
    }
    \newsavebox\ANDbox
    \sbox\ANDbox{$|$}

	% NOTE: HEADERS
    \begin{header}
        \fontsize{25 pt}{25 pt}\selectfont Yuhao Han (Harry)

        \vspace{5 pt}

        \normalsize
        \kern 5.0 pt%
		\mbox{\href{mailto:s2162783@ed.ac.uk}{\underline{s2162783@ed.ac.uk}}}%
        \kern 5.0 pt%
        \AND%
        \kern 5.0 pt%
		\mbox{\href{tel:+44 7769466855}{\underline{07769466855}}}%
        \kern 5.0 pt%
        \AND%
        \kern 5.0 pt%
		% \mbox{\href{https://yourwebsite.com/}{\underline{yourwebsite.com}}}%
        % \kern 5.0 pt%
        % \AND%
        % \kern 5.0 pt%
        % \mbox{\href{https://linkedin.com/in/yourusername}{linkedin.com/in/yourusername}}%
        % \kern 5.0 pt%
        % \AND%
        % \kern 5.0 pt%
		\mbox{\href{https://github.com/harryhanYuhao}{\underline{github.com/harryhanYuhao}}}%
    \end{header}

	% NOTE: INTRODUCTION
    \vspace{5 pt - 0.3 cm}
        \vspace{0.5 cm}
        \begin{onecolentry}
			I am a final year undergraduate student at the University of Edinburgh studying pure mathematics. 
			I am intrigued by the mathematical world, especially by geometry and algebra. 
		\\
        \vspace{0.15 cm}
			These subjects I believe will have imminent connections to the real world and will solve difficult problems that are otherwise unsolvale.
			To discover these connection requires ingenuities more than mathematics. 
			With such aspiring hopes, I am preparing myself for the furture enterprise by studying CS, chemistry, physics, and founding Yetin LTD.
        \end{onecolentry}
        % \vspace{0.2 cm}

	% NOTE: SECTION EDUCATION
    \section{Education}
        
        \begin{twocolentry}{
            Sept 2021 – Now
        }
            \textbf{University of Edinburgh}, BSc (Hons) Mathematics. Graduating in Jun 2025.
		\end{twocolentry}

        \vspace{0.10 cm}
        \begin{onecolentry}
            \begin{highlights}
                \item Expected Grade: First-Class Honours
                \item Modules taken: Algebra, Analysis, Differential Equation, Topology, Geometry, Lie Group, and Statistics
            \end{highlights}
        \end{onecolentry}

    
    \section{Experience}

		%%%% EXPERIENCE YETIN
        \begin{twocolentry}{
            Dec 2023 – Now
        }
			\href{https://yetin.net}{\textbf{\underline{Yetin Ltd}}}, cofounder
		\end{twocolentry}
        \vspace{0.10 cm}
        \begin{onecolentry}
            \begin{highlights}
				\item Yetin is a private limited company registered in Scotland, founded by me and another classmate.
				\item \href{https://book.yetin.net}{\underline{Yetin Guide for Coding}} is
					a tutorial for people trying to learn coding by themselves. 
					This book, emphasising on Linux and foundations of CS, adoptes an idiosyncratic approach and is prereleased on \href{https://book.yetin.net}{\underline{book.yetin.net}}.
            \end{highlights}
        \end{onecolentry}
		%%%% END OF EXPERIENCE YETIN

		%%%% SPACE BETWEEN EXPERIENCE LISTINGSS
        \vspace{0.2 cm}

		%%%% EXPERIENCE PCIE NTB
        \begin{twocolentry}{
            Oct 2023 – Feb 2024
        }
            \textbf{Research Dolphin Interconnect PCIe NTB}
		\end{twocolentry}
        \vspace{0.10 cm}
        \begin{onecolentry}
            \begin{highlights}
                \item I was part of the research group of Dr. Antonio Barbalace in the school of informatics at the University of Edinburgh.
				\item My work includes benchmarking virtual cachable memory provided by Dolphin NTB and developping drivers and softwares to increase its efficacy.
				All works were within Linux kernel.
				\item Part of the larger research project on Computer Express Link.
            \end{highlights}
        \end{onecolentry}
		%%%% END OF EXPERIENCE PCIE NTB

		%%%% SPACE BETWEEN EXPERIENCE LISTINGSS
        \vspace{0.2 cm}

		%%%% Experience Math Research
		\begin{twocolentry}{
	            May 2024 – Aug 2024
	        }
				\textbf{Summer Research on Algebraic Geometry}, Scottish Summer Research Scheme 
			\end{twocolentry}
	        \vspace{0.10 cm}
	        \begin{onecolentry}
	            \begin{highlights}
	                \item Study a specific type of affine hypertoric variety and develop an algorithm to count its $F_q$-count polynomials, which is compared with the h-polynomial of the complex associated with the corresponding hyperplane arrangement
					\item The theoretical algebraic problem was broken down to an equivalent combinatoric problem of counting points of intersections of simple hyperplane arrangements in $F^n_q$, $q$ is prime.
					\item Supervised by Dr. Sukjoo Lee in the University of Edinburgh.
	            \end{highlights}
	        \end{onecolentry}
		%%%% End of Experience Math Research
    
    \section{Bursary}
        
        \begin{samepage}
            \begin{twocolentry}{
                Jun 2024
            }
                Professor Sheridan's Simons Investigator Award, 2000£
            \end{twocolentry}

        %     \vspace{0.10 cm}
        %     
        %     \begin{onecolentry}
        %         \mbox{Frodo Baggins}, \mbox{\textbf{\textit{John Doe}}}, \mbox{Samwise Gamgee}
        %
        %         \vspace{0.10 cm}
        %         
        % \href{https://doi.org/10.1109/TASC.2023.3340648}{10.1109/TASC.2023.3340648}
        % \end{onecolentry}
        \end{samepage}


    
    \section{Projects}

        \begin{twocolentry}{
            Aug 2024
		}
		\href{https://github.com/harryhanYuhao/lox-rust.git}{\underline{\textbf{Lox Rust}}}
		\end{twocolentry}
        \vspace{0.10 cm}
        \begin{onecolentry}
            \begin{highlights}
                \item Fully functional interpreter from scratch in 5000 lines of Rust.
				It implements a mock language, lox.
            \end{highlights}
        \end{onecolentry}

        \vspace{0.15 cm}

        \begin{twocolentry}{
				July 2023
        }
		\href{https://github.com/harryhanYuhao/aformatter}{\underline{\textbf{A formatter}}}
		\end{twocolentry}
        \vspace{0.10 cm}
        \begin{onecolentry}
			A x86-64 assembly formatter in C.
        \end{onecolentry}


	% NOTE: SECTION PROGRAMMING TECHNOLOGIES
	\section{Programming Technologies}
        \begin{onecolentry}
			Linux \textbullet{} C \textbullet{} Rust \textbullet{} C++ \textbullet{} Bash \textbullet{} Fortran \textbullet{} \LaTeX \textbullet Git \textbullet Vim \textbullet Typescript \textbullet React \textbullet Nextjs \textbullet Java
			OpenGL \textbullet Lua \textbullet{} Julia \textbullet Python \textbullet{} x86-64 assembly \textbullet{} MATLAB \textbullet{} Unity \textbullet AWS

        \end{onecolentry}

        \vspace{0.2 cm}

	%NOTE: SECTION LANGUAGES
	\section{Languages}
	\begin{onecolentry}
		\textbf{Chinese}, Native
		\textbullet{} 
		\textbf{English}, TOEFL 115/120 
		\textbullet{} 
		\textbf{Latin}, Fluent to Read
		\textbullet{} 
		\textbf{Italian}, B1
		\textbullet{}
		\textbf{Russian}, Basic 
	\end{onecolentry}

\end{document}
