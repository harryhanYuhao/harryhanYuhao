\documentclass{article}

% Top margin, right margin, left margin, bottom margin, footnote skip
\usepackage[utf8]{inputenc}
\usepackage{biblatex}
\addbibresource{./references.bib}
% linktocpage shall be added to snippets.
\usepackage{hyperref,theoremref}
\hypersetup{
	colorlinks, 
	linkcolor={red!40!black}, 
	citecolor={blue!50!black},
	urlcolor={blue!80!black},
	linktocpage % Link table of content to the page instead of the title
}

\usepackage{blindtext}
\usepackage{titlesec}
\usepackage{amsthm}
\usepackage{thmtools}
\usepackage{amsmath}
\usepackage{amssymb}
\usepackage{graphicx}
\usepackage{titlesec}
\usepackage{xcolor}
\usepackage{multicol}
\usepackage{hyperref}
\usepackage{import}

%\DeclareMathOperator{\ker}{Ker}

% New Commands
\title{Letter of Motivation}
\author{Yuhao Han} 
\date{\today}

\begin{document}
\maketitle
% \tableofcontents

My primary reason for applying for the Mathematics (M.Sc.) at the University of Bonn is that I believe the rigorous training at Bonn will prepare me well for my future research in mathematics.
My long-term goal is to pursue a PhD degree in mathematics and become a mathematics researcher. 
My current knowledge in mathematics is inadequate for independent research, and I do not know which specific field I would like to specialize in.
Therefore, I believe further study at Bonn will be a necessary step to improve my mathematical ability and help me find a potential research topic.

My interest in mathematics lies in various branches of algebra, including algebraic geometry, algebraic topology, Lie groups, and their applications in other fields such as physics and computer science.
At my undergraduate institute, I have taken many of the foundation courses in pure mathematics. 
Currently at Oxford, I am taking more advanced courses, including homological algebra, algebraic topology, algebraic geometry, scheme theory, and computational topology.

I have conducted a summer research project on algebraic geometry at the University of Edinburgh, where I studied a specific type of affine hypertoric variety and developed an algorithm to count its $F_q$-count polynomials, which was compared with the $h$-polynomial of the complex associated with the corresponding hyperplane arrangement.
The theoretical algebraic problem was broken down into an equivalent combinatorial problem of counting points of intersections of simple hyperplane arrangements in $F^n_q$.
 
Currently, I am also working on an AI project exploring the mathematical foundation of deep learning from the perspective of neural collapse. Neural collapse is a phenomenon observed in many language models, where the last-layer features and paremeters converges to the highly symmetric equiangular tight frame (ETF) structure, which has deep roots in algebra and combinatorics.
I am trying to use the rich mathematical structure of ETF to explain the foundamental problem of why neural networks are so effective.
My collaboration on this topic with Zhikang Chen, a DPhil student at the University of Oxford, has produced a preprint paper: \textit{A Brain-Inspired Continual Learning System for Cross-User and Cross-Device Biosignals}.
I believe this direction of using geometry to explain deep learning, instead of the traditional statistical methods is promising, and I wish to continue working on this field.

As a sample of my mathematical writing, I uploaded my work submitted for the course Homological Algebra at Oxford in Spring 2026.
This work contains some advanced topics in cotriple homology and Hochschild homology.
Although most of the theorems and definitions are standard results, many of the proofs are original, which I believe can reflect my current mathematical ability.

At last, I shall explain why I choose the University of Bonn.
The Mathematics (M.Sc) at Bonn offers a wide range of rigorous and advanced courses in pure mathematics, which can be hardly rivaled by any other university in the world. 
Moreover, Bonn has a strong research group with many distinguished researchers in algebraic geometry and topology, which aligns well with my research interests.
I believe I will greatly benefit from the challenging courses and the stimulating academic atmosphere at Bonn.

\end{document}
