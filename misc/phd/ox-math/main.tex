\documentclass{article}

% Top margin, right margin, left margin, bottom margin, footnote skip
\usepackage[utf8]{inputenc}
\usepackage{biblatex}
\addbibresource{./references.bib}
% linktocpage shall be added to snippets.
\usepackage{hyperref,theoremref}
\hypersetup{
	colorlinks, 
	linkcolor={red!40!black}, 
	citecolor={blue!50!black},
	urlcolor={blue!80!black},
	linktocpage % Link table of content to the page instead of the title
}

\usepackage{blindtext}
\usepackage{titlesec}
\usepackage{amsthm}
\usepackage{thmtools}
\usepackage{amsmath}
\usepackage{amssymb}
\usepackage{graphicx}
\usepackage{titlesec}
\usepackage{xcolor}
\usepackage{multicol}
\usepackage{hyperref}
\usepackage{import}

%\DeclareMathOperator{\ker}{Ker}

% New Commands
\title{Statement of Purpose \\ \large For Oxford DPhil in Mathematics \\ Word Count: 775}
\author{Yuhao Han} 
\date{\today}

\begin{document}
\maketitle
% \tableofcontents

Pursuing a DPhil in Mathematics is a significant commitment. Therefore, I would like to use this document to explain my motivation, my relevant research experiences, and outline a potential research topic.

\subsection*{Motivation}

My greatest motivation to pursue a DPhil in mathematics is I my belief that mathematics beautiful. 
For example, group theory, whose formulation consists of four concise axioms, is able to describe all the symmetries in the physical world.
Discovering these rules is difficult. 
Yet the beauty of mathematics is enhanced, instead of hampered, by its difficulty. 
When contemplating a difficult mathematical problem, one is excited by the hope of solving it eventually.
This feeling will be elevated to a metaphysical state, with the realization that through the effort a new rule about the universe may be discovered. 

Among many institutions for DPhil studies in mathematics, I especially wish to study at Oxford.
My preference is not only because I am already a student here and love Oxford's academic environments, but also because I admire Oxford's mathematical research.
I especially admire the work of Prof. Christopher Douglas on framed combinatorial topology, which I believe is also relevant to my proposed research topic.


\subsection*{Research Experience} 

I have conducted a summer research project on algebraic geometry, where I studied a specific type of affine hypertoric variety and developed an algorithm to count its $F_q$-count polynomials, which was compared with the $h$-polynomial of the complex associated with the corresponding hyperplane arrangement.
The theoretical algebraic problem was broken down into an equivalent combinatorial problem of counting points of intersections of simple hyperplane arrangements in $F^n_q$.
 
Currently, I am also working on an AI project exploring the mathematical foundation of deep learning from the perspective of neural collapse. Neural collapse is a phenomenon observed in many language models, where the last-layer features and paremeters converges to the highly symmetric equiangular tight frame (ETF) structure, which has deep roots in algebra and combinatorics.
I am trying to use the rich mathematical structure of ETF to explain the foundamental problem of why neural networks are so effective.
My collaboration on this topic with Zhikang Chen, a DPhil student at the University of Oxford, has produced a preprint paper: \textit{A Brain-Inspired Continual Learning System for Cross-User and Cross-Device Biosignals}.
I believe this direction of using geometry to explain deep learning, instead of the traditional statistical methods is promising, and I wish to continue working on this field.

\subsection*{Geometric Methods for Determining Quantum Circuit Complexity}

In this section, I outline another of my potential research topics.
Although this problem is about quantum computing, its formulation and, likely its solution, will be purely mathematical.

An important yet unsolved problem in quantum computing is to determine the minimal number of gates that can be used to approximate any given unitary operation.
Nielsen presented the following geometric approach \cite{Nielsen-geometryquantumcomputation, nielsen2005, Nielsen2006}.

Given an operator $U \in SU(2^n)$, suppose it is generated by Hamiltonian $H(t)$ satisfying the Schrödinger equation $\frac{dU}{dt} = -iHU$.
$H(t)$ has a Pauli operator expansion $H = \sum_{\sigma}h_{\sigma}\sigma + \sum_{\tau}h_{\tau}\tau$, where $\sigma$ ranges over one and two body interactions, $\tau$ ranges over three or more body interactions, and $h_{\sigma}, h_{\tau}$ are real numbers. 
By choosing an appropriate penalty, $p$, we can define a metric whose components are
$$
	g_{\sigma \tau} = \begin{cases}
		0 \text{ if } \sigma \neq \tau \\
		1 \text{ if } \sigma  = \tau \text{ is one or two-body interactions} \\
		p^2 \text{ otherwise}
	\end{cases}
$$

Let $d(I, U)$ denote the length of the minimal geodesic on this metric between $U$ and the identity on $SU(2^n)$. 
As $SU(2^n)$ is connected, such a geodesic must exist.
Nielsen showed that $d(I,U)$, up to a constant, gives a lower bound on the number of one or two-body gates to precisely generate $U$ \cite{nielsen2005}.

I found Nielsen's geometric approach illuminating, and would like to expand his theory.
One direction is to develop a method to find the minimal geodesic on $SU(2^n)$ between $I$ and any $U$.
This question, however, is likely to be classical intractable \cite{Nielsen-geometryquantumcomputation}. 
Yet symmetries may reduce this intractable problem to a simpler setting. 
There are at least two possible approaches:
\begin{enumerate}
	\item We may quotient $SU(2^n)$ by permutation group, representing the change of qubit indices. 
		This approach was explored by recent research on permutation invariant quantum circuit \cite{mansky2023permutationinvariantquantumcircuits} 
	\item Since we are mostly interested in algorithms that exhibit quantum supremacy, we may reduce $SU(2^n)$ by quotienting the Clifford group, as elements of the Clifford group can be simulated efficiently on classical computers \cite{gottesman1997, gottesman1998}.
\end{enumerate}

If the minimal geodesic has been found, \textit{is it possible to construct the minimal circuit to generate $U$, given the minimal geodesic $d(I, U)$?} 
Nielsen had hinted its possibility without further elaboration. 
I believe the geodesic will at least give some clues to the construction.

This problem is essentially geometric.
The potential solution to the problem will likely depend on advanced theories of Riemannian geometry, algebraic topology, Lie groups, and Lie algebras.
It is possible that analytical solutions are extremely difficult. 
In such a case, however, I believe meaningful numerical solutions can be found, by innovations in numerical algebra, geometry, and topology.
In particular, I would like to explore its connection with combinatorial topology, such as the framed combinatorial topology developed by Prof Christopher Douglas.

\printbibliography
\end{document}
