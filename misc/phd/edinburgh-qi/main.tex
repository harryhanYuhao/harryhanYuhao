\documentclass{article}

% Top margin, right margin, left margin, bottom margin, footnote skip
\usepackage[utf8]{inputenc}
\usepackage{biblatex}
\addbibresource{./references.bib}
% linktocpage shall be added to snippets.
\usepackage{hyperref,theoremref}
\hypersetup{
	colorlinks, 
	linkcolor={red!40!black}, 
	citecolor={blue!50!black},
	urlcolor={blue!80!black},
	linktocpage % Link table of content to the page instead of the title
}

\usepackage{blindtext}
\usepackage{titlesec}
\usepackage{amsthm}
\usepackage{thmtools}
\usepackage{amsmath}
\usepackage{amssymb}
\usepackage{graphicx}
\usepackage{titlesec}
\usepackage{xcolor}
\usepackage{multicol}
\usepackage{hyperref}
\usepackage{import}

%\DeclareMathOperator{\ker}{Ker}

% New Commands
\title{Statement of Intent \\ \large Submitted for EPSRC Application}
\author{Yuhao Han} 
\date{\today}

\begin{document}
\maketitle
% \tableofcontents

This short passage aims to explain my motivation to pursue a PhD degree in quantum informatics, why I am particularly interested in EPSRC program, and how my knowledge and past experiences make me a suitable candidate for this program.

My primary motivation to study quantum computing is that I believe it has the potential to bring revolutionary changes to our society, just as classical computers have done in the last several decades. 
However, the field of quantum computing is vast and advanced, involving deep knowledge in physics, computer science, and mathematics.
Without the valuable knowledge and research experiences gained through a PhD degree, one may not be able to make any meaningful contributions.
I am excited to contribute to the development of quantum computing and through my efforts bring positive impacts to our society. 
Therefore, pursuing a PhD degree in quantum informatics becomes a natural choice for me.

The theoretical aspects of quantum informatics are largely built upon advanced mathematical ideas, such as differential geometry, Lie group, and topology. 
This leads to my second motivation: I enjoy studying mathematics, and I am particularly interested in the mathematical foundation of quantum informatics.
Studying mathematics is an enjoyable task to me, and I can not be more satisfied if, by doing so, I can also bring meaningful impacts to the real world by studying quantum informatics.

Not only have I clear motivations to study quantum informatics, my interdisciplinary skills in mathematics, computer science, and physics developed from my past education and research experiences also make me a suitable candidate.
I have received extensive training in mathematics and mathematical physics which are relevant to quantum informatics. 
During my undergraduate studies at Edinburgh, I took modules on differential geometry, analysis, algebraic topology, Lie group, and general relativity. 
Currently at Oxford, I am taking category theory and homological algebra, while auditing lecture courses on atomic clocks and quantum matters.
At the same time, I am also a self-taught programmer. 
I am fluent to code in C, C++, Python, and Rust. 
I have comprehensive knowledge in complexity theory, algorithms, compilers and operation systems. 

It would be helpful to further elucidate my past research experiences. 
In the final year at Edinburgh, I completed a group project with two other classmates titled \textit{Quantifying Chaos}. 
Our work attempted to define chaos in dynamical systems with rigour, with discussion of logistic bifurcations and a description of the surprising universal Feigenbaum constants. 
We made two original contributions in this project.
Our first contribution was to devise novel efficient algorithms using C++ GMP arbitrary precision library for numerical simulations of chaos, which required a higher precision than the IEEE double precision floating point format provides. 
We have also combined our knowledge in pure mathematics, especially in topology and analysis, with dynamical systems to come up with some original proofs on properties of bifurcations and chaos. 
One of our graders had commented that \textit{This project shows some original content and is well beyond the scope of a graduate course in dynamical system.}

Besides my undergraduate final year project, I also have several other research experiences. 
I have conducted a summer research on algebraic geometry, where I studied the $F_p$-polynomial of a specific type of affine hypertoric varieties by counting intersections of simple hyperplanes in $F_p^n$.
I have also worked in the research group led by Dr. Antonio Barbalace at the University of Edinburgh, where I was responsible for developing Linux kernel drivers for Dolphin interconnect PCIe NTB, as part of a greater project on computer express links (CXL). 
Currently, I am working on an AI project exploring the mathematical foundation of supervised and reinforcement learning. 
My work explores the connection between the ideas of Free Energy Principle and the surprising observation of neural collapse.
My collaboration with Zhikang Chen, a DPhil candidate at the University of Oxford has produced a preprint paper: \textit{A Brain-Inspired Continual Learning System for Cross-User and Cross-Device Biosignals}.

The EPSRC CDT program is particularly appealing to me for several reasons. 
Firstly, although I have some research experiences and knowledge in quantum informatics, I am not prepared to conduct independent research. Therefore, the first year's training that covers advanced topics in mathematics, physics, and computer science provided by EPSRC will be extremely beneficial to me.
Secondly, I believe EPSRC’s strong industrial connections and entrepreneurship program aligns well with my goal to provide positive societal impacts through research.
Lastly, I have previously studied at Edinburgh and I greatly enjoy the city and the Scottish culture.
It would be my great pleasure to study in Edinburgh again.

In summary, I believe my knowledge and past research experiences would make me a suitable candidate for EPSRC CDT, and I am confident that the training, research experience, and professional development offered by the EPSRC CDT would provide an ideal environment for my growth as a quantum informatics researcher.


\end{document}
