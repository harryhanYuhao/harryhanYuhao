\documentclass{article}

\usepackage[tmargin=3.5cm,rmargin=4cm,lmargin=4cm,bmargin=3cm]{geometry} 
% Top margin, right margin, left margin, bottom margin, footnote skip
\usepackage[utf8]{inputenc}


\usepackage{hyperref}
\hypersetup{
	colorlinks, 
	linkcolor={red!40!black}, 
	citecolor={blue!50!black},
	urlcolor={blue!80!black},
	linktocpage % Link table of content to the page instead of the title
}
\usepackage[backend=biber, natbib=true]{biblatex}
\addbibresource{./references.bib}

% linktocpage shall be added to snippets.
\usepackage{xcolor}
\usepackage{graphicx}
\usepackage{blindtext}
\usepackage{amsthm}
\usepackage{amsmath}
\usepackage{amssymb}

\renewcommand{\P}{\textbf{P}}

\title{A Survey of Potential Research Interest \\ \large Served as the Research Proposal for Warwick Mathematics CDT}
\author{Yuhao Han} 
\date{\today}

\begin{document}
\maketitle

My interests in mathematics mostly lie in topology, geometry, and their intersections with computer science. 
At the moment, I do not have a clear vision on what specific subject I shall study for PhD, so I will instead describe one potential research problem I found interesting and one of my past research experiences. 

\subsection*{Geometric Methods for Determining Quantum Circuit Complexity}

An important and unsolved problem in quantum computing is to determine the minimal number of gates that can be used to approximate a given unitary operations.
Nielsen presented the following geometric approach \cite{Nielsen-geometryquantumcomputation, nielsen2005, Nielsen2006}.

Given an operator $U \in SU(2^n)$, suppose it is generated by Hamiltonian $H(t)$ satisfying the Schrödinger equation $\frac{dU}{dt} = -iHU$, with the requirement that $U(t_f) = U$. 
$H(t)$ has a Pauli operator expansion $H = \sum_{\sigma}h_{\sigma}\sigma + \sum_{\tau}h_{\tau}\tau$, where $\sigma$ ranges over one and two body interactions, $\tau$ ranges over three or more body interactions, and $h_{\sigma}, h_{\tau}$ are real numbers. 
By choosing an appropriate penalty, $p$, a metric on $SU(2^n)$, can be defined, whose components are
$$
	g_{\sigma \tau} = \begin{cases}
		0 \text{ if } \sigma \neq \tau \\
		1 \text{ if } \sigma  = \tau \text{ is one or two-body interactions} \\
		p^2 \text{ otherwise}
	\end{cases}
$$

Let $d(I, U)$ denote the length of the minimal geodesic given by this metric between $U$ and the identity on $SU(2^n)$. 
As $SU(2^n)$ is connected, such a geodesic must exist.
Nielsen showed that $d(I,U)$, up to a constant, gives a lower bound to the number of one or two-body gates to precisely generate $U$ \cite{nielsen2005}, and any $U$ can be approximated with $d(I, U)$ number of gates with arbitrary precision up to a polynomial factor \cite{Nielsen2006}.

This geometric perspective is powerful, because it enables us to study $SU(2^{n})$ using the vast repertoire of Riemannian Geometry.
For example, we may imitate the approach of General Relativity by studying connections, Christoffel symbols, and geodesic differential equation \cite{Nielsen2006}.

The natural next step to expand Nielsen's theory is to find the minimal geodesic on $SU(2^n)$ between $I$ and any $U$.
This problem, however, is likely to be classically intractable \cite{Nielsen-geometryquantumcomputation}. 
Therefore, I propose to consider this problem: \textit{Is there a general method to find the minimal geodesic on a reduction of $SU(2^n)$?} 
There are two immediate potential reductions of $SU(2^n)$ that may be of interest
\begin{enumerate}
	\item We may quotient $SU(2^n)$ by a subgroup of symmetry.
		One candidate for symmetry is permutations. 
		Its plausibility is strengthened by recent research on permutation invariant quantum circuit \cite{mansky2023permutationinvariantquantumcircuits} and the fact that reordering of qubits shall have no physical effects on quantum computers.
	\item Since we are mostly interested in algorithms that exhibit quantum supremacy, we may reduce $SU(2^n)$ by quotienting the Clifford group. It is well known that elements of Clifford group can be simulated efficiently on classical computers \cite{gottesman1997, gottesman1998}.
		Since Clifford group is a Lie subgroup which fits into our theory nicely, our geometric method may reveal additional insights about the origin of quantum supremacy.
\end{enumerate}

If the minimal geodesic has been found, I further ask: \textit{is it possible to construct the minimal circuit to generate $U$, given the minimal geodesic $d(I, U)$?} 
Nielsen has hinted its possibility, without any further investigations \cite{Nielsen2006}. 
I believe the geodesic will at least give some hints of the construction, as it specifies a continuous trajectory form $U$ to $I$, and each point on the trajectory is a unitary operation.

This problem is essentially geometric, although it originated from quantum computing. 
The potential solution to the problem will likely depend on advanced theory of Riemannian geometry, Lie group, and Lie algebras.

\subsection*{Past Project: Bifurcations and Feigenbaum Constants in Chaotic Dynamical Systems}

During the final year of my undergraduate degree at the University of Edinburgh, I completed a group research project on logistic bifurcation and the universality of Feigenbaum constants. 
Most of our work was a survey of existing literature, but our proofs and numerical analysis are original. 
Here is a brief summary of our work.

Logistic map \eqref{logistic} is a quadratic polynomial depending on one parameter, $\lambda$.
\begin{equation}\label{logistic}
	L_{\lambda}(x) = 4 \lambda x(1-x)	
\end{equation}
When $0 \leq \lambda \leq 1$, the logistic map is concave downward, unimodal (has a unique maximum), and maps $[0,1]$ to $[0,1]$. 
Therefore, we may define a dynamical system $X$, whose value at discrete time $t \in \mathbb{N}^{+}$ is defined iteratively as $x_{t+1} = L_{\lambda}(x_t)$.

$X$ exhibits some surprising behaviours listed below and demonstrated in Figure \ref{fig:Demonstration of Bifurcations}, 

\begin{enumerate}
	\item For $0 < \lambda < a_0 = \frac{3}{4}$, the system has a unique fixed point, i.e., a stable 1-orbit.
		This 1-orbit bifurcates to two orbits, which further bifurcates to $2^2, 2^3, \cdots, 2^n $ orbits as $\lambda$ increases.
	\item For any stable $2^n$ orbit, there exists $\lambda$ such that $0.5$ is part of the orbit. Label this $\lambda$ as $A_n$. 
		There exists a surprising limit, called Feigenbaum $\delta$ constant,
		$$
			\delta = \lim_{n \to \infty} \frac{A_{n-1} - A_{n-2}}{A_n - A_{n-1}} \approx 4.669201
		$$
	At each $A_i$, let $d_i$ denote the distance between $0.5$ and the nearest point in the $2^i$ orbit. 
	The ratio of $d_i$ also converges to a constant, called Feigenbaum $\alpha$ constant,
		$$
			\alpha = \lim_{n \to \infty} \frac{d_n}{d_{n+1}} \approx 2.502911
		$$
	\item Bifurcation does not exhausts the whole spetrum for $\lambda$. After some $\lambda$, the system does not have any apparent stable orbit and appears chaotic.
\end{enumerate}

The bifurcations and existence of Feigenbaum constants were studied in detail by Feigenbaum \cite{F1}, who further showed that these phenomena are universal in a large class of functions which are unimodal, concave downward, and satisfy some very weak regularity conditions.

We have made several original contributions, including

\begin{enumerate}
	\item We use Devaney's topological definition of chaos \cite{Devaney_green_book_chaos_definition} to rigorously prove that the logistic map is chaotic at $\lambda = 1$. (The phenomenum is well known, but our proof is original)
	\item Inspired by Feigenbaum's renormalisation methods, we demonstrated the existence of periodic doubling bifurcations using elementary ideas in analysis and topology.
	\item Detailed calculations of Feigenbaum constants were performed using original algorithms implemented in C++.
\end{enumerate}


\begin{figure}[htbp]
	\centering
	\includegraphics[width=\textwidth]{./demonstration of feigenbaum constants.png}
	\caption{Demonstration of Bifurcations. Left: Bifurcation diagram of logistic map. This graph is produced in the following way: For each $\lambda$, we iterate $L_{\lambda}$ for 1000 times, and plot the last 100 iterations as a faint blue dot.
	Right: the same bifurcation diagram with logarithmic scale to demonstrate the details of bifurcations. $A_n$ are points $\lambda$ such that $0.5$ is part of a $2^n$ orbit. 
	$d_n$ are the distances between $0.5$ and the nearest point in the $2^n$ orbit.
	}
	\label{fig:Demonstration of Bifurcations}
\end{figure}


\newpage

\printbibliography
\end{document}
